\documentclass[12pt, letterpaper]{article}

\usepackage{float}
\usepackage{cprotect}
\usepackage{array}
\usepackage{hyperref}

% Metadata
\title{Specification for Web API testing language}

\author{Jakab Zeller}

\date{October 2021}

\begin{document}

\maketitle

\section{Tokens}

The following are all the tokens, along with their regular expressions, that the lexical analyser accepts.

\begin{center}
    \begin{verbatim}
        # Tokens passed to the parser
        Number    = `[-+]?(\d*\.)?\d+'
        StringLit = `"(\\"|[^"])*"'
        Ident     = `[a-zA-Z_]\w*'
        Method    = `(GET|HEAD|POST|PUT|DELETE|CONNECT|OPTIONS|TRACE|PATCH)'
        While     = `while'
        For       = `for'
        Do        = `do'
        Break     = `break'
        Then      = `then'
        End       = `end'
        Function  = `function'
        Return    = `return'
        If        = `if'
        Elif      = `elif'
        Else      = `else'
        In        = `in'
        True      = `true'
        False     = `false'
        Null      = `null'
        Punct     = `[-,()*/+{};&!=:<>]|\[|\]'
        # Ignored tokens
        comment    = `//.*|/\*.*?\*/'
        whitespace = `\s+'
    \end{verbatim}
\end{center}

All capitalised symbols are tokens that are passed from the lexical analyser to the parser. Whereas, the uncapitalised symbols are read by the lexical analyser and ignored by the parser.

\section{Grammar}

\begin{center}
    \begin{verbatim}
        Program  -> Block
        Block    -> { Stmt } [ RetStmt ]
        Stmt -> (   JSONPath "=" Exp
                  | FuncCall
                  | MethCall
                  | Break
                  | While Exp Do Block End
                  | For Ident "=" Exp ";" Exp [ ";" Exp ] Do Block End
                  | For Ident [ "," Ident ] In Exp Do Block End
                  | Function JSONPath FuncBody
                  | If Exp Then Block { ElifSeg } [ ElseSeg ] End )? ";"

        # Moved to seperate rules for readability
        ElifSeg  -> Elif Exp Then Block
        ElseSeg  -> Else Block

        # JSON Path
        JSONPath -> Part { "." Part }
        Part     -> Ident { "[" Exp "]" }

        RetStmt  -> Return [ Exp ] [ ";" ]
        FuncCall -> JSONPath Args
        MethCall -> Method Args
        FuncBody -> "(" [Params] ")" Block End
        Params   -> Ident { "," Ident }
        Args     -> "(" [ExpList] ")"
        ExpList  -> Exp { "," Exp }

        # Our "arithmetic expressions" have 5 levels of precedence
        # "||": lowest precedence
        Exp      -> Prec5T { Prec5 }
        Prec5    -> Prec5Op Prec5T Prec5
        Prec5Op  -> "||"

        # "&&": 2nd lowest precedence
        Prec5T   -> Prec4T { Prec4 }
        Prec4    -> Prec4Op Prec4T Prec4
        Prec4Op  -> "&&"

        # "==" and "!=": 3rd lowest precedence
        Prec4T   -> Prec3T { Prec3 }
        Prec3    -> Prec3Op Prec3T Prec3
        Prec3Op  -> "==" | "!="

        # "<", ">", "<=", and ">=": 4th lowest precedence
        Prec3T   -> Prec2T { Prec2 }
        Prec2    -> Prec2Op Prec2T Prec2
        Prec2Op  -> "<" | ">" | "<=" | ">="

        # "+" and "-": 5th lowest precedence
        Prec2T   -> Prec1T { Prec1 }
        Prec1    -> Prec1Op Prec1T Prec1
        Prec1Op  -> "+" | "-"

        # "*" and "/": Highest precedence
        Prec1T   -> Factor { Prec0 }
        Prec0    -> Prec0Op Factor Prec0
        Prec0Op  -> "*" | "/"

        # The factors are our base values
        Factor   -> Null |
                    False |
                    True |
                    Number |
                    StringLit |
                    JSONPath |
                    JSON |
                    FuncCall |
                    MethodCall |
                    "(" Exp ")"

        # JSON literal
        JSON     -> Object | Array
        Object   -> "{" Members "}"
        Members  -> Pair { "," Pair }
        Pair     -> Key ":" Exp
        Key      -> StringLit | JSONPath | AriExp
        Array    -> "[" ExpList "]"
    \end{verbatim}
\end{center}

\section{Semantics}

\subsection{Statements}

Statements can be one of the following:

\begin{center}
    \begin{enumerate}
        \item An assignment. Only one assignment operator is supported: `='.
        \item A function call.
        \item The break keyword.
        \item A `while' loop. The condition of which is an expression. This is explained more in the \hyperref[sec:expressions]{`Expressions'} subsection.
        \item A `for' loop. Supports either the ``traditional" C-style format or the iterator style format. The iterator style `for' loop has an optional additional iterator assignment for key-value pair iteration or index-value iteration.
        \item A function definition.
        \item An `if-elif-else' block.
        \item A `NOP'. Defined by a statement followed by a singular semi-colon.
    \end{enumerate}
\end{center}

\subsection{Variables and values}

All values are stored internally as JSON parsable strings. This means that the supported ``data types" are strings, floats, integers, arrays, objects, `true'/`false' and `null'. Some other semantics:

\begin{center}
    \begin{itemize} 
        \item Variables can be modified using `JSON path'.
        \item Variable assignment will carry out a \textbf{deep copy}.
        \item \textbf{Variables are always passed by reference}{Variables are always passed by reference} and expressions are passed by value.
        \item \textbf{Variables are declared and defined} when a new \textbf{root identifier} is used in a JSONPath assignment that \textbf{doesn't exist} in the \textbf{current scope}.
        \item \textbf{Variables are set} when the root identifier \textbf{exists} in the \textbf{current scope}.
        \item \textbf{Variables are filled sparsely} when a \textbf{JSONPath accessing a non-existent sublevel} of the JSON is assigned to. For instance:
        \begin{verbatim}
            hello_world = {
                "hello": "world"
            };
            hello_world.world[1].hello = "world";
            /*
            hello = {
                "hello": "world",
                "world": [
                    null,
                    {"hello": "world"}
                ]
            }
            */
        \end{verbatim}
        \item The resulting values of boolean operators are converted to the integers `0' or `1' when they are a part of a larger arithmetic expression.
        \item JSON literals can be directly assigned to variables.
    \end{itemize}
\end{center}

\subsection{JSON path}

The grammar for the language supports rudimentary JSON path expressions. This includes the standard `.' property accessing or the string indexing style. Below are some valid JSON path expressions:

\begin{figure}[H]
    \begin{center}
        \begin{verbatim}
            json[0].path.is.cool
            json[0]["path"].is.["cool"]
            json[x + 2]["path"].is.["co" + "ol"]
        \end{verbatim}
    \end{center}
    \vspace{-1.5em}
    \cprotect\caption{Examples of some valid JSON paths. These all access the \verb|json| variable.}
\end{figure}

JSON paths can be assigned to. This sets the appropriate property of the variable.

\subsection{Expressions}
\label{sec:expressions}

Both \textbf{boolean and arithmetic expressions are a part of the same precedence hierarchy}. If binary expressions and arithmetic expressions are mixed together, the result of a boolean operator will be coerced to either a `0' or a `1'. \textbf{All operators are left-to-right associative}.

\subsubsection{Operator precedence}

\begin{figure}[H]
    \begin{center}
        \begin{tabular}{| m{2cm} | m{2cm} | m{5cm} |}
            \hline
            Precedence & Operators & Description\\
            \hline
            \textbf{0} & \verb|* /| & Multiplication and division\\
            \hline
            \textbf{1} & \verb|+ -| & Addition and subtraction\\
            \hline
            \textbf{2} & \verb|< > <= >=| & Less than, greater than, less than or equal to, and greater than or equal to\\
            \hline
            \textbf{3} & \verb|== !=| & Equal and not equal\\
            \hline
            \textbf{4} & \verb|&&| & Logical and\\
            \hline
            \textbf{5} & \verb+||+ & Logical or\\
            \hline
        \end{tabular}
    \end{center}
    \caption{Operators and their precedence from highest precedence (top) to lowest precedence (bottom)}
\end{figure}

\subsubsection{Conditions}

In statements that require conditions (such as `if-elif-else', `while', and `for' statements) a purely arithmetic, a purely binary operator expression, or a mixed expression can all be used. Here is a list of all truthy and falsy values.

\begin{figure}[H]
    \begin{center}
        \begin{tabular}{| c | c | c |}
            \hline
            Type & Truthy value & Falsy value\\
            \hline
            Boolean & \verb|true| & \verb|false|\\
            \hline
            null & & \verb|null|\\
            \hline
            Number & \verb|!= 0| & \verb|0|\\
            \hline
            String & \verb|"..."| & \verb|""|\\
            \hline
            JSON & \verb|{...}| & \verb|{}|\\
            \hline
        \end{tabular}
    \end{center}
    \caption{Table of truthy and falsy values of each type}
\end{figure}

% \begin{center}
%     \begin{verbatim}
%         chunk ::= block
%         block ::= {stat} [retstat]
%         stat ::=  ‘;’ | 
%              varlist ‘=’ explist | 
%              functioncall | 
%              label | 
%              break | 
%              goto Name | 
%              do block end | 
%              while exp do block end | 
%              repeat block until exp | 
%              if exp then block {elseif exp then block} [else block] end | 
%              for Name ‘=’ exp ‘,’ exp [‘,’ exp] do block end | 
%              for namelist in explist do block end | 
%              function funcname funcbody | 
%              local function Name funcbody | 
%              local attnamelist [‘=’ explist] 
%         attnamelist ::=  Name attrib {‘,’ Name attrib}
%         attrib ::= [‘<’ Name ‘>’]
%         retstat ::= return [explist] [‘;’]
%         label ::= ‘::’ Name ‘::’
%         funcname ::= Name {‘.’ Name} [‘:’ Name]
%         varlist ::= var {‘,’ var}
%         var ::=  Name | prefixexp ‘[’ exp ‘]’ | prefixexp ‘.’ Name 
%         namelist ::= Name {‘,’ Name}
%         explist ::= exp {‘,’ exp}
%         exp ::=  nil | false | true | Numeral | LiteralString | ‘...’ |
%              functiondef | prefixexp | tableconstructor | exp binop exp |
%              unop exp 
%         prefixexp ::= var | functioncall | ‘(’ exp ‘)’
%         functioncall ::=  prefixexp args | prefixexp ‘:’ Name args 
%         args ::=  ‘(’ [explist] ‘)’ | tableconstructor | LiteralString 
%         functiondef ::= function funcbody
%         funcbody ::= ‘(’ [parlist] ‘)’ block end
%         parlist ::= namelist [‘,’ ‘...’] | ‘...’
%         tableconstructor ::= ‘{’ [fieldlist] ‘}’
%         fieldlist ::= field {fieldsep field} [fieldsep]
%         field ::= ‘[’ exp ‘]’ ‘=’ exp | Name ‘=’ exp | exp
%         fieldsep ::= ‘,’ | ‘;’
%         binop ::=  ‘+’ | ‘-’ | ‘*’ | ‘/’ | ‘//’ | ‘^’ | ‘%’ | 
%              ‘&’ | ‘~’ | ‘|’ | ‘>>’ | ‘<<’ | ‘..’ | 
%              ‘<’ | ‘<=’ | ‘>’ | ‘>=’ | ‘==’ | ‘~=’ | 
%              and | or
%         unop ::= ‘-’ | not | ‘#’ | ‘~’
%     \end{verbatim}
% \end{center}

% Program  -> Block.
% Block    -> Stmt | Stmt RetStmt.
% Stmts    -> Stmt Stmt |.
% Stmt     -> semi |
%     JSONPath equal Exp |
%     FuncCall |
%     Break |
%     While Exp Do Block End |
%     For Ident equal Exp semi Exp Do Block End |
%     For Ident equal Exp semi Exp semi Exp Do Block End |
%     For Ident In Exp Do Block End |
%     For Ident comma Ident In Exp Do Block End |
%     Function JSONPath FuncBody |
%     If Exp Then Block Elifs End |
%     If Exp Then Block Elifs Else Block End.
% Elifs    -> Elif Exp Then Block Elifs |.
% JSONPath -> Part Parts.
% Parts    -> dot Part Parts |.
% Part     -> Ident Indices.
% Indices  -> sqrL Exp sqrR Indices|.
% RetStmt  -> Return |
%             Return semi |
%             Return Exp semi |
%             Return Exp.
% FuncCall -> JSONPath Args.
% FuncBody -> parenL Params parenR Block End | parenL parenR Block End.
% Params   -> Ident Idents.
% Idents   -> comma Ident Idents |.
% Args     -> parenL ExpList parenR | parenL parenR.
% Exp      -> Null |
%             False |
%             True |
%             Number |
%             StringLit |
%             JSONPath |
%             JSON |
%             AriExp.
% ExpList  -> Exp Exps.
% Exps     -> comma Exp Exps |.

% AriExp   -> Prec5T Precs5.
% Precs5   -> Prec5 Precs5|.
% Prec5    -> Prec5Op Prec5T Prec5.
% Prec5Op  -> union.

% Prec5T   -> Prec4T Precs4.
% Precs4   -> Prec4 Precs4|.
% Prec4    -> Prec4Op Prec4T Prec4.
% Prec4Op  -> and.

% Prec4T   -> Prec3T Precs3.
% Precs3   -> Prec3 Precs3|.
% Prec3    -> Prec3Op Prec3T Prec3.
% Prec3Op  -> eq | ne.

% Prec3T   -> Prec2T Precs2.
% Precs2   -> Prec2 Precs2|.
% Prec2    -> Prec2Op Prec2T Prec2.
% Prec2Op  -> lt | gt | lte | gte.

% Prec2T   -> Prec1T Precs1.
% Precs1   -> Prec1 Precs1|.
% Prec1    -> Prec1Op Prec1T Prec1.
% Prec1Op  -> plus | minus.

% Prec1T   -> Factor Precs0.
% Precs0   -> Prec0 Precs0|.
% Prec0    -> Prec0Op Factor Prec0.
% Prec0Op  -> mult | div.

% Factor   -> Number | JSONPath | parenL AriExp parenR.

% JSON     -> Object | Array.
% Object   -> curlyL Members curlyR.
% Members  -> Pair Pairs.
% Pairs    -> comma Pair Pairs |.
% Pair     -> Key colon Exp.
% Key      -> StringLit | JSONPath | AriExp.
% Array    -> sqrL ExpList sqrR.

\end{document}