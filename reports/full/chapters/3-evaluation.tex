\chapter{Evaluation}
\label{chap:evaluation}

Overall, I am quite pleased with what I accomplished in the given amount of time. If I had planned anymore features, or changed any of the technologies that I had used, I think I would have been struggling to finish the project in time. I would like to continue iterating on the \verb|sttp| language in the future.

\section{What went well}

The project continued at a steady pace with no real hitches that affected the overall running of the project.

I had some apprehensions of the performance of the \verb|batch| statement implementation. However, I knew that using Go to implement the interpreter would give the best shot at acquiring the required threaded performance.

\verb|sttp| proved to be useful during the rewrite of the tests that were mentioned in \hyperref[chap:motivations]{motivations}. The tests that check if the scraping targets have changed on the sources that I scrape from are stored within a single directory. This directory is then executed as a \verb|TestSuite| daily. If there are unforeseen changes the \verb|TestSuite| will fail, indicating that there has been a change, and then the appropriate changes can be made to the scrape procedures themselves.

\section{Things that I would change}

\cprotect\subsection{\verb|sttp| compiler?}

In the future I would like to create a frontend for a compiler that translates \verb|sttp| code into an intermediary format, such as Java bytecode or WASM, that can then have native code generated from it. Or, maybe a transpiler that compiles to another programming language such as Go or C. This isn't because the performance of the \verb|sttp| interpreter is unsatisfactory, I am just interested in doing so.

\cprotect\subsection{Syntactical changes}

\verb|sttp| uses operators to perform actions that would usually be assigned to a builtin function. Even though I think this methodology is quite interesting, it might lead to alienation amongst new users. Therefore, I think introducing builtin functions for these actions is more suitable. I would also make a few changes to the grammar of \verb|sttp|:

\begin{itemize}
    \item More assignment operators such as \verb|+=| and \verb|-=|.
    \item Remove semi-colons from the end of statements ending with \verb|end|. Even though this makes sense from a completeness standpoint (all statements should end with a `\verb|;|'), it is ultimately unnecessary.
    \item Use of \verb|MethodCall|s and \verb|FunctionCall|s as the root property of JSONPath expressions. This is so that the results of these don't first need to be stored within a variable to be accessed via JSONPath.
    \item Making most of the tokens matched by the lexer smaller, so that less typing needs to be done by the programmer and it is easier to fit onto a terminal screen.
    \item Creating a syntax highlighting extension for \href{https://code.visualstudio.com/api/language-extensions/syntax-highlight-guide}{vscode using TextMate} and for other editors.
    \item Not a grammatical change, but lazy operation of logical operators would reduce the overhead of long logical expressions.
\end{itemize}

Most of these decisions come after realising that the grammar of the language makes it too `verbose', and there is a lot of time spent writing keywords and tokens.

\cprotect\subsection{The \verb|batch| statement}

Another thing I would change is the algorithm used to compute the \verb|batch| statement. Since the implementation I have thought of several better, and more efficient ways of executing it. One of them involves starting the worker goroutines alongside the interpreter in the first pass. Whenever the interpreter steps into a \verb|MethodCall| AST node, it will enqueue it as a job instantly. This job will then be executed by a worker thread and then added to the result queue, ready for the second pass. The second pass would be unchanged from the original algorithm. This method leverages the speed increase from executing the HTTP requests in parallel, but is also doing the analysis pass (first pass) at the same time.

\cprotect\subsection{API documentation generation}

Most programming languages such as Java and Go, allow you to generate formal documentation for the source code from the documentation and comments imbedded within it. This is something that I would like to add to \verb|sttp|, but instead of documenting the implementation of the source code itself, the programmer would document the API. Such as the actions and resources available for the API, and the parameters accepted by each. This documentation could then generate a set of webpages that would serve as documentation for the API itself.

\cprotect\subsection{Environments}

REST/HTTP clients, such as Insomnia and Postman, allow the user to create multiple `environments' which store some global configurations about the version or type of the Web API being tested. For instance, you could create an environment to test the Web API when it is running locally, as well as another environment to test the deployed version of it. In \verb|sttp| I would imagine that these environments could be created as \verb|.json| files that are stored within a particular directory of the \verb|TestSuite|. Then when running the \verb|TestSuite| you would supply a flag indicating the name of the environment to use. Then, before running each script in the \verb|TestSuite|, each \verb|sttp| \verb|VM| would first have a variable, such as \verb|env|, set within the bottommost stack frame as global variable that contains the environment that was chosen.

\cprotect\subsection{The \verb|after-evaluation| branch}

I have already started on some of these changes within the `\verb|after-evaluation|' branch. It is not part of the submitted code as it changes quite a lot of the syntax/semantics of \verb|sttp| and these have not yet been reflected within the specification.
