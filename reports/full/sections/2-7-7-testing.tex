\cprotect\subsection{Test suites and the \verb|test| statement}

\begin{center}
    \textbf{Git commit}
    \begin{minted}{markdown}
Language: implementation of TestSuite capabilities + TestStatement
- Set up test data structures as well as interfaces (11/12/2021 - 02:50:34)
- TestStatement.Eval filled (11/12/2021 - 02:51:01)
- Methods for test datastructures (11/12/2021 - 02:51:14)
- Added TestSuite.Run + implemented TestStatement.Eval (12/12/2021 - 01:20:55)
- Added TestSuite testing capabilities (12/12/2021 - 01:21:14)
- Added an initial test suite test (12/12/2021 - 01:34:16)
- Created interfaces for TestResults so that they could be used within parser
  package (12/12/2021 - 01:34:50)
    \end{minted}
    \vspace{-1em}
    \tiny{December 12, 2021}
\end{center}

Test suites and \verb|test| statements are one of the most important features and motivations behind creating \verb|sttp|. They allow the programmer to construct test suites using a directory structure containing \verb|sttp| scripts. The \hyperref[sec:data-structures-vm]{VM section} mentions the \verb|TestResults| field that exists within the \verb|VM| interface's implementation. This is a pointer to an instance of the \verb|TestResults| type that has the following definition:

\inputminted[firstline=29, lastline=35, autogobble, breaklines, tabsize=4]{go}{../../src/test.go}

\verb|TestResults|' purpose is to store the results of all \verb|test| statements that have been evaluated by the interpreter thus far. \verb|TestStatement.Eval| will first create a new instance of \verb|TestResults| if there is not one already defined within the \verb|VM|. Then it will evaluate the expression that follows the \verb|test| statement, casting it to boolean if necessary. The result of this will then be added to the \verb|TestResults| using the \verb|TestResults.AddTest| method. This will append a \verb|TestResult| (definition below) to \verb|TestResults|.

\inputminted[firstline=21, lastline=27, autogobble, breaklines, tabsize=4]{go}{../../src/test.go}

Here, a pointer is stored to the \verb|TestStatement| that was evaluated for this result. This is in order to retrieve the position of the \verb|test| statement within the script (via \verb|TestStatement.GetPos|) and so that the \verb|test| statement can be `pretty-printed' (via \verb|TestStatement.String|).

When the interpreter is evaluating a directory structure of \verb|sttp| scripts, an instance of \verb|TestSuite| will be constructed:

\inputminted[firstline=88, lastline=98, autogobble, breaklines, tabsize=4]{go}{../../src/test.go}

The \verb|TestSuite| is a recursive data-structure, in that if a \verb|TestSuite| represents the root directory, then a new \verb|TestSuite| will be created for each directory (and directory's directory) within the root directory. Each \verb|TestSuite| level can also have a set of \verb|sttp| scripts, that can all produce their own \verb|TestResults|.

After an instance of \verb|TestSuite| has been constructed for a root directory structure of \verb|sttp| scripts, the \verb|TestSuite.Run| method can be run to evaluate all the \verb|sttp| scripts found. \verb|TestSuite.Run| will walk the directory structure recursively. Each time it encounters an \verb|sttp| file, it will create a \verb|VM| to evaluate it, storing the resulting \verb|TestResults| within the current \verb|TestSuite|. Otherwise, if the current file is a directory, a new \verb|TestSuite| will be created; inheriting the \verb|Config| from current \verb|TestSuite|. \verb|TestSuite.Run| will then be called recursively on this new \verb|TestSuite|. This new \verb|TestSuite| will be stored in the \verb|InnerSuites| field of the current \verb|TestSuite|. \verb|TestSuite| implements the \verb|Stringer| interface, so the \cprotect{\hyperref[sec:test-suites-and-the-test-statement-test-suite-output]}{`results' of the \verb|TestSuite| can be displayed back to the user}. However, one downside of doing it this way is that test results won't be displayed to the user in real-time.

Each \verb|TestSuite|, \verb|TestResults|, and \verb|TestResult| inherits a pointer to a \verb|TestConfig| instance. This is just a simple struct that stores some configuration flags on how the \verb|TestSuite| and \verb|test| statement should behave. Currently it only has one available flag: \verb|BreakOnFailure|. If \verb|BreakOnFailure| is set then if an expression following a \verb|test| statement evaluates to \verb|false|, \hyperref[sec:development-try-catch-throw-errors-context]{an uncatchable error will be thrown}; stopping the execution of \verb|VM.Eval| or \verb|TestSuite.Run|.

The source code for \verb|TestSuite|, \verb|TestResults|, and \verb|TestResult| can be found within the \verb|src/test.go| file. Whereas, the source code for the \verb|test| statement can be found in the \verb|src/parser/eval.go| file.
