\subsection{If-Elif-Else}

\textit{Examples can be found in: }\cprotect{\hyperref[appendix:sttp-examples-3]}{\verb|src/_examples/example_03/example_03.sttp|}

\begin{center}
    \textbf{Git commit}
    \begin{minted}[breaklines]{markdown}
Language: Heap rework + Symbol -> Value + IfElifElse
- Reworked Heap datastructure (06/12/2021 - 02:57:30)
- Renamed Symbol to Value (06/12/2021 - 02:57:54)
- Refactored a lot of files to accomodate these 
  changes (06/12/2021 - 02:58:14)
- Implemented ifelifelse Eval referrer and added example
  code to test it (06/12/2021 - 03:57:36)
    \end{minted}
    \vspace{-1em}
    \tiny{December 6, 2021}
\end{center}

The AST node/non-terminal representing \verb|IfElifElse| statements are constructed from an \verb|Expression| that acts as the condition for the first \verb|if|, which is followed by \verb|Block| to execute if the condition is truthy. \verb|elif| statements are then matched repeatedly until an optional \verb|else| token or the \verb|end| token is matched. The AST node is represented in code as follows.

\inputminted[firstline=304, lastline=320, autogobble, breaklines, tabsize=4]{go}{../../src/parser/ast.go}

The \verb|elif| statements are split into their own AST node for code readability. \verb|IfElifElse.Eval| is fairly straightforward in its implementation. One thing to note is the use of anonymous functions. In this case it is used to evaluated an \verb|Expression| and return a \mintinline[breaklines]{go}{bool}. This is used to evaluate the conditions within \verb|if| and \verb|elif|.

\inputminted[firstline=801, lastline=815, autogobble, breaklines, tabsize=4]{go}{../../src/parser/eval.go}
